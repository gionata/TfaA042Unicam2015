\documentclass[italian]{beamer}

\usepackage{default}
\usepackage[T1]{fontenc}
\usepackage[utf8x]{inputenc}
\usepackage[italian]{babel}
%\usepackage{microtype}


\mode<presentation>{%
		\AtBeginSection{%
		\begin{frame}
			\frametitle{Piano della presentazione}
			\tableofcontents[currentsection]
		\end{frame}%
}}

\title[Privacy \& Social Network]{ {\Large Privacy \& Social Network\\}{\small per gli studenti}}
\author[G. Massi]{Gionata Massi}
\institute[IIS Lamport--Knuth]{IIS Lamport Knuth}
\date[Camerino, 18 Apr. 2015]{Camerino -- 18 aprile 2015}

\begin{document}
	
\begin{frame}
	\titlepage
\end{frame}

\begin{frame}{Piano della presentazione}
	\tableofcontents
\end{frame}

\section{Obiettivi formativi}

\begin{frame}{Obiettivi formativi}
	\begin{enumerate}
		\item comprendere il concetto di privacy
		\item riconoscere l'importanza della tutela della privacy
		\item conoscere e saper tutelare i propri diritti digitali
		\item riconoscere ed evitare i pericoli dei social network
		\item valutare i rischi connessi alla pubblicazione dei propri dati personali su internet e i social network
	\end{enumerate}	
\end{frame}


\section{Introduzione}

\subsection{I grandi fratelli\ldots e dati spiati}

\begin{frame}{Grandi fratelli e dati rubati (spiati)}
	
	\begin{block}{Il \textit{Grande Fratello} -- \textbf{George Orwell}, \emph{``1984''}}
		\begin{itemize}
			\item Echelon: telecom;
			\item PRISM: dati dai Big IT;
			\item leggi sulla privacy delle ICT company degli USA.
		\end{itemize}
	\end{block}
	 
\end{frame}

\begin{frame}{Fatti: sistemi di controllo}
	Greenwald ha rivelato che la National Security Agency (NSA) statunitense avrebbe a disposizione delle vere e proprie \alert<2>{``backdoor''} all’interno dei server delle maggiori società IT del mondo (cd BIG IT) quali Google, facebook, Yahoo ecc. {[}\ldots{]} tuttavia {[}\ldots{]}\ il sistema {[}\ldots{]} riuscirebbe a spiare il traffico di dati a prescindere da particolari accessi a backdoor, visto che normalmente gli utenti della rete lasciano, legalmente, tantissime tracce delle loro attività online. {[}\ldots{]} \alert<3>{tutto il traffico Internet veicola per gli USA}, tutti coloro che utilizzano o hanno utilizzato Facebook, Google, Yahoo e così di seguito \alert<3>{potrebbero essere stati ``controllati''}{[}\ldots{]}.
\end{frame}


\begin{frame}{Edward Snowden}
	
	Un ex tecnico della Cia con gravi problemi giudiziari (WikiLeaks). Latitante.
	
	\begin{block}{Una sua citazione}
		Avevo una vita comoda: ragazza, lavoro e carriera. Ma ho deciso di sacrificare tutto perché non avevo la coscienza a posto nel permettere che il governo Usa distruggesse ogni privacy, libertà della rete, e diritti fondamentali delle persone in tutto il mondo
	\end{block}
\end{frame}

\subsection{I social network\ldots e dati sensibili divulgati inconsapevolmente}

\begin{frame}{I Social network}
	\begin{itemize}
		\item Permettono uno spiccato livello di interazione tra sito web e utente
		\item Ognuno condivide un po’ di se stesso portando propri pensieri, fotografie, filmati
		\item Implicano un progressiva perdita della propria privacy
	\end{itemize}
\end{frame}

\begin{frame}{I Social Network e la condivisione spontanea}
	\begin{block}{Condivisione spontanea sui social network}
		\begin{itemize}
			\item Quali dati personali ho esposto pubblicamente?
			\item Chi possiede i dati che mi riguardano?
			\item Posso eliminarli o riapproriarmene?
			\item Qual \'e la mia identit\`a digitale?
		\end{itemize}
	\end{block}
\end{frame}

\begin{frame}{Qualcuno ha reale interesse per i miei dati?}
	\begin{itemize}
		\item molte applicazioni, legate al gioco e/o all'intrattenimento, prelevano dati, spesso all'insaputa degli utenti, per poi utilizzarli sia a fini pubblicitari sia per un controllo sull'identità degli utenti
		\item informazioni riguardanti la sfera privata di un soggetto vengano divulgate a terzi senza il preventito consenso dell’interessato (foto taggate, post sui propri profili\ldots)
		\item richieste di dati agli Internet Service Provider (ISP)
		\item un social network vale (economicamente) per quanti dati possiede
	\end{itemize}
\end{frame}

\begin{frame}{E se cancellassi il mio profilo dal social network?}
	Il più delle volte accade che persino la cancellazione dal servizio, ovvero l'eliminazione dei propri dati, non corrisponda ad una completa rimozione dei contenuti immessi dallo stesso utente, il che fa si che informazioni riguardanti la sfera privata di un soggetto vengano divulgate a terzi senza il preventito consenso dell'interessato
\end{frame}

\begin{frame}{Conseguenze sociali}
	\begin{block}{Manipolazione}
		\begin{itemize}
			\item Le pubblicit\`a mirate e il profilamento a fini commerciali (BigG, FB\ldots)
		\end{itemize}
	\end{block}
	\begin{block}{Discriminazione}
		\begin{itemize}
			\item La mia scheda politica/religiosa/sanitaria
			\item Le mie bravate
			\item Confini labili fra vita reale e virtuale
		\end{itemize}
	\end{block}
\end{frame}

\section{Il codice della privacy: D.Lgs. n. 196/2003}

\subsection{Finalit\`a e terminologia}

\begin{frame}{Finalit\`a}
	\begin{block}{Finalit\`a}
		Garantire il trattamento dei dati personali
	\end{block}
\end{frame}

\begin{frame} %[allowframebreaks]
	{Trattamento dei dati personali}
	%\begin{block}{Trattamento dei dati personali}
	\begin{columns}
		\begin{column}{0.45\textwidth}
		\begin{itemize}
			\item raccolta
			\item registrazione
			\item organizzazione
			\item conservazione
			\item consultazione
			\item elaborazione
			\item blocco
			\item modificazione
		\end{itemize}
		\end{column}
		\begin{column}{0.45\textwidth}
		\begin{itemize}
			%\framebreak			
			\item utilizzo
			\item interconnessione
			\item comunicazione
			\item diffusione
			\item cancellazione
			\item distruzione
			\item selezione
			\item estrazione
			\item raffronto
		\end{itemize}
	\end{column}
	\end{columns}
	%\end{block}		
\end{frame}

\section{Discussione}

\begin{frame}{Cosa ne penso?}
	\begin{itemize}
		\item comprendere il concetto di privacy
		\item riconoscere l'importanza della tutela della privacy
		\item conoscere e saper tutelare i propri diritti digitali
		\item riconoscere ed evitare i pericoli dei social network
		\item valutare i rischi connessi alla pubblicazione dei propri dati personali su internet e i social network
	\end{itemize}		
\end{frame}

\end{document}
